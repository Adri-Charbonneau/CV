%%%%%%%%%%%%%%%%%
% This is an sample CV template created using altacv.cls
% (v1.4, 12 Apr 2021) written by LianTze Lim (liantze@gmail.com). Now compiles with pdfLaTeX, XeLaTeX and LuaLaTeX.
%
%% It may be distributed and/or modified under the
%% conditions of the LaTeX Project Public License, either version 1.3
%% of this license or (at your option) any later version.
%% The latest version of this license is in
%%    http://www.latex-project.org/lppl.txt
%% and version 1.3 or later is part of all distributions of LaTeX
%% version 2003/12/01 or later.
%%%%%%%%%%%%%%%%

%% Use the "normalphoto" option if you want a normal photo instead of cropped to a circle
% \documentclass[10pt,a4paper,normalphoto]{altacv}

\documentclass[10pt,a4paper,ragged2e,withhyper]{altacv}
%% AltaCV uses the fontawesome5 and packages.
%% See http://texdoc.net/pkg/fontawesome5 for full list of symbols.

% Change the page layout if you need to
\geometry{left=0.5cm,right=0.5cm,top=0.5cm,bottom=0.5cm,columnsep=1cm}

% The paracol package lets you typeset columns of text in parallel
\usepackage{paracol}

% Change the font if you want to, depending on whether
% you're using pdflatex or xelatex/lualatex
\ifxetexorluatex
  % If using xelatex or lualatex:
  \setmainfont{Roboto Slab}
  \setsansfont{Lato}
  \renewcommand{\familydefault}{\sfdefault}
\else
  % If using pdflatex:
  \usepackage[rm]{roboto}
  \usepackage[defaultsans]{lato}
  % \usepackage{sourcesanspro}
  \renewcommand{\familydefault}{\sfdefault}
\fi

% Change the colours if you want to
\definecolor{GreyTitle}{HTML}{777777}
\definecolor{SlateGrey}{HTML}{2E2E2E}
\definecolor{LightGrey}{HTML}{666666}
\definecolor{Greentitle}{HTML}{4cae4f}
\definecolor{PastelRed}{HTML}{f79a4d}
\definecolor{Greyrule}{HTML}{E6E6E6}
\colorlet{name}{GreyTitle}
\colorlet{tagline}{Greentitle}
\colorlet{heading}{Greentitle}
\colorlet{headingrule}{Greyrule}
\colorlet{subheading}{PastelRed}
\colorlet{accent}{PastelRed}
\colorlet{emphasis}{SlateGrey}
\colorlet{body}{LightGrey}

% Change some fonts, if necessary
\renewcommand{\namefont}{\Huge\rmfamily\bfseries}
\renewcommand{\personalinfofont}{\footnotesize}
\renewcommand{\cvsectionfont}{\LARGE\rmfamily\bfseries}
\renewcommand{\cvsubsectionfont}{\large\bfseries}


% Change the bullets for itemize and rating marker
% for \cvskill if you want to
\renewcommand{\itemmarker}{{\small\textbullet}}
\renewcommand{\ratingmarker}{\faCircle}

%% Use (and optionally edit if necessary) this .cfg if you
%% want to use an author-year reference style like APA(6)
%% for your publication list
% \input{pubs-authoryear.cfg}

%% Use (and optionally edit if necessary) this .cfg if you
%% want an originally numerical reference style like IEEE
%% for your publication list
% \input{pubs-num.cfg}

%% sample.bib contains your publications
% \addbibresource{sample.bib}

\begin{document}
\name{Adrien CHARBONNEAU}
\tagline{Naturaliste écologue et ornithologue passionné}
%% You can add multiple photos on the left or right
\photoR{3cm}{PROFIL-RE.jpg}
% \photoL{2.5cm}{Yacht_High,Suitcase_High}

\personalinfo{%
  % Not all of these are required!
  \email{adrien.charbonneau@free.fr}
  \phone{06 52 28 85 25}
  \car{Véhiculé}
  \mailaddress{19 rue Antoine Primat, 69100 Villeurbanne}
  % \location{Location, COUNTRY}
  \homepage{www.adriencharbonneau.fr}
  % \twitter{@twitterhandle}
  \linkedin{adrien-charbonneau}
  \git{Adrien}
  \researchgate{Adrien-Charbonneau}
  %% You can add your own arbitrary detail with
  %% \printinfo{symbol}{detail}[optional hyperlink prefix]
  % \printinfo{\faPaw}{Hey ho!}[https://example.com/]
  %% Or you can declare your own field with
  %% \NewInfoFiled{fieldname}{symbol}[optional hyperlink prefix] and use it:
  % \NewInfoField{gitlab}{\faGitlab}[https://gitlab.com/]
  % \gitlab{your_id}
}

\makecvheader
%% Depending on your tastes, you may want to make fonts of itemize environments slightly smaller
% \AtBeginEnvironment{itemize}{\small}

%% Set the left/right column width ratio to 6:4.
\columnratio{0.55}

% Start a 2-column paracol. Both the left and right columns will automatically
% break across pages if things get too long.
\begin{paracol}{2}
\cvsection{Expériences professionnelles}

\cvevent{Chargé d'études ornithologiques}{Synergis Environnement - Agence Est}{Janvier 2021 -- En cours}{Vaulx-en-Velin (69)}
Inventaire avifaune (IPA, migration, rapaces,...) + faune généraliste / suivi mortalité / VNEI / Impacts et Mesures

\divider

\cvevent{Impact de la pollution lumineuse sur la biodiversité}{P.N.R des Baronnies provençales - IMBE}{Février 2020 -- Août 2020}{Drôme (26) - Hautes-Alpes (05)}
Bio-écoacoustique / Pollution lumineuse / Activité et indices acoustiques

\divider

\cvevent{Compétition entre deux espèces de Geckos}{Agence Française pour la Biodiversité - P.N. de Port-Cros}{Septembre 2019}{Île de Porquerolles (83)}
Étude scientifique / Milieu insulaire / Compétition écologique

\divider

\cvevent{Caractérisation du territoire de reproduction de la \\Pie-grièche méridionale}{Ligue pour la Protection des Oiseaux - PACA}{Mars 2019 -- Juin 2019}{Bouches-du-Rhône (13)}
Capture / Radiopistage / Territoires de reproduction / Publication


\divider

\cvevent{Lien entre communauté d’Orthoptères et hauteur de végétation}{Institut Méditerranéen de Biodiversité et d'Écologie - IMBE}{Septembre 2018}{Vallée de l'Ubaye (04)}
Étude scientifique : protocole - terrain - analyse / Travail en équipe


\divider

\cvevent{Chargé d'études faunistiques}{Alp'Pagès / Ecoscim}{Mai 2018 -- Août 2018}{France métropolitaine}
Alpes / Vosges / Pyrénées / Inventaire / SIG / VNEI

\switchcolumn

\cvsection{Études}

\cvevent{Master en Biodiversité, Écologie, Évolution}{Université Aix-Marseille}{Sept 2018 -- Sept 2020}{Marseille (13)}
%%Ajout de détails

\divider

\cvevent{Licence en Biologie générale}{Université Grenoble-Alpes}{Sept 2015 -- Mai 2018}{Grenoble (38)}
%%Ajout de détails

\divider

\cvevent{Baccalauréat Général Scientifique}{Institution Notre-Dame}{2015}{Valence (26)}
%%Ajout de détails

\cvsection{Langues et informatique}

\cvevent{Langues vivantes}{}{}{}
\begin{itemize}
\item Anglais : Courant (B1)
\item Espagnol : Notions (A2)
\end{itemize}

\divider

\cvevent{Informatique et technologies}{}{}{}
\begin{itemize}
\item Analyse de données : R, QGIS, MARK, MAXENT
\item Bureautique : Pack Office, LaTex
\item C2i : obtention en 2017
\item Site internet : \url{https://www.adriencharbonneau.fr}
\end{itemize}


% use ONLY \newpage if you want to force a page break for
% ONLY the current column
% \newpage


%% Switch to the right column. This will now automatically move to the second
%% page if the content is too long.


\cvsection{Compétences}

\cvtag{Ornithologie}
\cvtag{Naturalisme}
\cvtag{Écologie}\\
% \divider\smallskip
\cvtag{Conservation et gestion}
\cvtag{Analyse de données}\\
\cvtag{Cartographie}
\cvtag{Bio-Éco-Acoustique}\\
\cvtag{Science}
\cvtag{Rédaction}
\cvtag{Aisance orale}

\end{paracol}
\begin{paracol}{1}
\bigskip
\cvsection{Centres d'intérêt et autres activités}

\cvachievement{\faBinoculars}{Naturalisme}{Suivis migration / Wetlands / STOC / EPOC / Suivi amphibiens / Vigie-chiro  / Participation Atlas de Biodiversité - Régionaux}

\divider

\cvachievement{\faLeaf}{Bénévolat/Adhésion}{Ligue pour la Protection des Oiseaux (2012) / Société Nationale de la Protection de la Nature (2019) / Sympetrum (2020) / Alauda (2021)}

\divider

\cvachievement{\faSearch}{Sciences participatives}{Saisies d’observations sur les bases de données Biolovision (Faune-France / Naturalist) / Telabotanica / Spipoll}

\end{paracol}


\end{document}